\documentclass{ctexbeamer}

% packages
% \usepackage{ctex}
\usepackage{latexsym,amsmath,multicol,booktabs,calligra}
\usepackage{graphicx}

% colors
\usepackage{xcolor}
\definecolor{deepblue}{rgb}{0,0,0.5}
\definecolor{deepred}{rgb}{0.6,0,0}
\definecolor{deepgreen}{rgb}{0,0.5,0}
\definecolor{halfgray}{gray}{0.55}

% fonts
\usefonttheme[stillsansserifsmall]{serif}
\usepackage{xeCJK}

% code style
\usepackage{listings}
\lstset{
    basicstyle=\ttfamily\small,
    keywordstyle=\bfseries\color{deepblue},
    emphstyle=\ttfamily\color{deepred},    % Custom highlighting style
    stringstyle=\color{deepgreen},
    numbers=left,
    numberstyle=\small\color{halfgray},
    rulesepcolor=\color{red!20!green!20!blue!20},
    frame=shadowbox,
}

% info
\title[甘雨快来我身边] % (optional, use only with long paper titles)
{There Is No Largest Prime Number}

\subtitle
{Include Only If Paper Has a Subtitle}

\author[Author] % (optional, use only with lots of authors)
{LuoLuoKong}

\institute[理学院] % (optional, but mostly needed)
{
  冯家屯男子电刀大学
 }

\date[CFP 2003] % (optional, should be abbreviation of conference name)
{Conference on Fabulous Presentations, 2003}

\subject{Theoretical Computer Science}

\logo{\vspace*{-0.5cm}\includegraphics[height=1cm]{pic/logo.png}}

\usepackage{jk} % basic

% main
\begin{document}

  % cover
  \begin{frame}
    \titlepage
    \begin{figure}[htpb]
      \begin{center}
          \vspace{-0.5cm}
          \includegraphics[width=0.2\linewidth]{pic/pink.jpg}
      \end{center}
    \end{figure}
  \end{frame}
  
  % table-of-contents
  \begin{frame}
    \frametitle{Outline}
    \tableofcontents
  \end{frame}
  
  % pages
  \section{One}
    \subsection{One-one}
    \begin{frame}{Theorem and Example}
      \begin{theorem}
          我是李方琦爸爸。
      \end{theorem}
      \begin{example}
          我也是。
      \end{example}
    \end{frame}
    
    \subsection{One-two}
    \begin{frame}{items & enumerate}
        \begin{itemize}
            \item 和泉纱雾
            \item 远坂凛
        \end{itemize}
        \vbox{}
        \begin{enumerate}
            \item 空条徐伦
            \item 明日香·兰格雷
        \end{enumerate}
    \end{frame}
  
  \section{Two}
    \subsection{Two-one}
    \begin{frame}{Proof & block}
      \begin{proof}
          \begin{enumerate}
              \item<1-> First, we need;
              \item<2-> Secound, we can;
              \item<3-> Then, take it easy;
              \item<1-> Yes, we do it!
          \end{enumerate}
      \end{proof}  
    \end{frame}
    
    \subsection{Two-two}
    \begin{frame}
        \begin{columns}[t]
            \column{.5\textwidth}
            \begin{block}{Answered Questions}
                我能吃下玻璃而不伤身
            \end{block}
    
            \column{.5\textwidth}
            \begin{block}{Open Questions}
                i can eat windows.
            \end{block}
        \end{columns}
    \end{frame}
  
  \section{code like}
  \subsection{python}
  \begin{frame}[fragile]{Code}
      \begin{lstlisting}
        import re
        li = input()
        PhoneRegex = re.compile(r'xxx')
        m = re.search(li)
      \end{lstlisting}
  \end{frame}
  
  \subsection{模板对比}
  \begin{frame}{冯家屯电刀 vs 北京大学}

\begin{figure}[htpb]
    \centering
    \includegraphics[width=0.2\linewidth]{pic/logo.jpg}
\end{figure}

\vspace{-0.6cm}

\begin{table}
\centering
\begin{tabular}{ccc} 
\toprule
特征      & 冯家屯电刀 & 北京大学       \\ 
\hline
小狐狸logo & \checkmark    & \times      \\ 
\hline
\fbox{稳重的矩形风格} & \checkmark    & \times  \\ 
\hline
{\color{blue}略带忧伤的蓝色主题} & \checkmark     & \times  \\ 
\hline
{\color{cyan}支}{\color{ceruleanblue}持}{\color{cherryblossompink}超}{\color{coquelicot}多}{\color{cornflowerblue}颜}{\color{daffodil}色} & \checkmark    & \times   \\
\bottomrule
\end{tabular}
\end{table}
  
  \vspace*{0.5cm}
  \par\centering
  总结:冯家屯电刀 {\Huge\alert{赢} !}
      
\end{frame}
  
  % thanks!
  \begin{frame}
  \begin{center}
      {\Huge\kaishu 感谢!}
  \end{center}
  \end{frame}
\end{document}